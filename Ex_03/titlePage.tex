

\subsection*{students handing in this solution set}

\begin{tabular*}{\textwidth}{l@{\extracolsep{\fill}}lll}
  \toprule
  last name & first name & student ID & enrolled with \\
  \midrule
  \midrule
  Nikolskyy
  & Oleksander
  & TODO
  & TODO
  \\
  Schier
  & Marie
  & TODO
  & TODO
  \\
  Doll
  & Niclas
  & 3075509
  & Uni Bonn
  \\
  Safavi
  & Arash
  & TODO
  & TODO
  \\
  Wani
  & Mohamad Saalim
  & TODO
  & TODO
  \\
  Bonani
  & Mayara Everlim
  & TODO
  & TODO
  \\
  \bottomrule
\end{tabular*}
\newpage

%%%%%
%%%%% DO NOT EDIT THE FOLLOWING
%%%%%

\subsection*{general remarks}

As you know, your instructor is an avid  proponent of open science and education. Therefore, \textbf{MATLAB implementations will not be accepted} in this course.

The goal of this exercise is to get used to scientific Python. There are numerous resources on the web related to Python programming. Numpy and Scipy are well documented and Matplotlib, too, comes with numerous tutorials. Play with the code that is provided. Most of the above tasks are trivial to solve, just look around for ideas as to how it can be done.

Remember that you have to achieve at least 50\% of the points of the exercises to be eligible to the written exam at the end of the semester. Your grades (and credits) for this course will be decided based on the exam only, but --once again-- you have to succeed in the exercises to get there.
  
Your solutions have to be \emph{satisfactory} to count as a success. Your code and results will be checked and need to be convincing.

If your solutions meets the above requirements and you can demonstrate that they work in practice, it is a \emph{satisfactory} solution.

A \emph{very good} solution (one that is rewarded full points) requires additional efforts especially w.r.t. to readability of your code. If your code is neither commented nor well structured, your solution is not good! The same holds for your discussion of your results: these should be concise and convincing and demonstrate that you understood what the respective task  was all about. Striving for very good solutions should always be your goal!
 



\subsection*{practical advice}



The problem specifications you'll find below assume that you use python / numpy / scipy for your implementations. They also assume that you have imported the following
\begin{python}
import numpy as np
import numpy.linalg as la
import matplotlib.pyplot as plt
\end{python}