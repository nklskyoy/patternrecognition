
\subsection*{task 3.1 [5 points] \\[1ex] whitened data is of unit covariance}

In lecture 08, we briefly looked at data pre-processing techniques such as \emph{data whitening}. Recall that, if we are given an $m \times n$ data matrix of real valued vectors
\begin{equation*}
\mat{X} = 
\begin{bmatrix}
\vdash    & \vdash    &        & \vdash \\
\vec{x}_1 & \vec{x}_2 & \cdots & \vec{x}_n \\
\vdash    & \vdash    &        & \vdash \\
\end{bmatrix}
\end{equation*}
then the process of whitening these data consists of the following three steps
\begin{enumerate}
\item compute the transformation
  \begin{equation*}
  \mat{Y} = \mat{X} \, \bigl( \mat{I} - \tfrac{1}{n} \, \opt{\vec{1}}{\vec{1}} \bigr)
  \end{equation*}
\item determine the spectral decomposition
  \begin{equation*}
  \tfrac{1}{n} \, \opt{\mat{Y}}{\mat{Y}} = \mat{U} \mat{\Lambda} \, \trn{\mat{U}}
  \end{equation*}
\item  compute the transformation
  \begin{equation*}
  \mat{Z} = \mat{U} \, \mat{\Lambda}^{-\frac{1}{2}} \trn{\mat{U}} \mat{Y}
  \end{equation*}
\end{enumerate}
Also recall that we claimed the covariance matrix of the data in $\mat{Z}$ to be the identity matrix. Your task is now to prove this claim, i.e. prove that
\begin{equation*}
\frac{1}{n} \, \opt{\mat{Z}}{\mat{Z}} = \mat{I}
\end{equation*}
(Remember that $\mat{U}$ is orthogonal and $\mat{\Lambda}$ is diagonal \ldots)
\color{blue} \\[1ex]
%%%%%
%%%%%
%%%%% enter your proof here
%%%%%
%%%%%
\begin{align*}
\opt{\mat{Z}}{\mat{Z}} 
&= (\mat{U} \mat{\Lambda}^{-\frac12} \mat{U}^T \mat{Y}) (\mat{U} \mat{\Lambda}^{-\frac12} \mat{U}^T \mat{Y})^T \\
&= \mat{U} \mat{\Lambda}^{-\frac12} \mat{U}^T (\mat{Y} \mat{Y}^T ) \mat{U} \mat{\Lambda}^{-\frac12} \mat{U}^T \\
&= \mat{U} \mat{\Lambda}^{-\frac12} \mat{U}^T (n \cdot \frac1n \mat{Y} \mat{Y}^T ) \mat{U} \mat{\Lambda}^{-\frac12} \mat{U}^T \\
&= \mat{U} \mat{\Lambda}^{-\frac12} \mat{U}^T (n \cdot \mat{U} \mat{\Lambda} \mat{U}^T ) \mat{U} \mat{\Lambda}^{-\frac12} \mat{U}^T \\
&= n \cdot \mat{U} \mat{\Lambda}^{-\frac12} \mat{U}^T \mat{U} \mat{\Lambda} \mat{U}^T \mat{U} \mat{\Lambda}^{-\frac12} \mat{U}^T \\
&= n \cdot \mat{U} \mat{\Lambda}^{-\frac12} \mat{\Lambda} \mat{\Lambda}^{-\frac12} \mat{U}^T \\
&= n \cdot \mat{U} \mat{U}^T \\
&= n \cdot \mat{I}
\end{align*}

%%%%%
%%%%%
%%%%%
%%%%%
%%%%%
\color{black}




