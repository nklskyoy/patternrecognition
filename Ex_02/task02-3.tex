
\subsection*{task 2.3 [5 points] \\[1ex] complex numbers as matrices}

Python has an in-built data type for complex numbers. For instance, the two numbers $z_1 = 3 \cdot 1 + 4 \cdot i$
and $z_2 = 2 \cdot 1 - 2 \cdot i$
%\begin{align*}
%z_1 & = 3 \cdot 1 + 4 \cdot i \\
%z_2 & = 2 \cdot 1 - 2 \cdot i
%\end{align*}
can be implemented as
\begin{python}
z1 = complex(3, +4)
z2 = complex(2, -2)
\end{python}
or simply as
\begin{python}
z1 = 3 + 4j
z2 = 2 - 2j
\end{python}
\textbf{NOTE:} For people other than electrical engineers, it may be confusing that Python refers to the imaginary unit $i$ as \texttt{j}. But it is how it is.

Here is your \emph{first sub-task}: using Python, compute the values of the four simple expressions $z_1 + z_2$, $z_1 \cdot z_2$, $z_1^* = 3 \cdot 1 - 4 \cdot i$, and $\lvert z_1 \rvert$. 

Now consider this: complex numbers can also be represented in terms of real valued $2 \times 2$ matrices. To see this, consider the two ``basis'' matrices
\begin{align*}
\mat{1} & = 
\begin{bmatrix} 
+1 &  0 \\ 
 0 & +1 
\end{bmatrix} \\[1ex]
\mat{i} & = 
\begin{bmatrix} 
 0 & -1 \\ 
+1 &  0 
\end{bmatrix} 
\intertext{and let}
\mat{z}_1 & = 3 \, \mat{1} + 4 \, \mat{i} \\
\mat{z}_2 & = 2 \, \mat{1} - 2 \, \mat{i}
\end{align*}

Here is your \emph{second sub-task}: implement these objects in Numpy and compute $\mat{z}_1 + \mat{z}_2$, $\mat{z}_1 \cdot \mat{z}_2$, $\trn{\mat{z}_1}$, and $\sqrt{\det{\mat{z}_1}}$ (where $+$ and $\cdot$ denote matrix addition and multiplication).
%%%%%
%%%%%
%%%%% enter your code into the following environment
%%%%%
%%%%%
\begin{python}
# subtask_1
z1 = complex(3, +4)
z2 = complex(2, -2)
print("Z1+Z2 = ", z1 + z2)
print("Z1.Z2 = ", z1 * z2)
print("Z1* = ", z1.conjugate())
print("|Z1| = ", abs(z1))

# subtask_2
identity = np.identity(2)
i = np.array([[0, -1], [1, 0]])
z1 = np.array(3 * identity + 4 * i)
z2 = np.array(2 * identity - 2 * i)
print("Z1+Z2:")
print(z1 + z2, end='\n\n')
print("Z1*Z2:")
print(z1 @ z2, end='\n\n')
print("Transpose(Z1):")
print(z1.transpose(), end='\n\n')
print("Sqrt(Det(Z1)):")
print(np.sqrt(np.linalg.det(z1)), end='\n\n')

\end{python}

%%%%%
%%%%%
%%%%%
%%%%%
%%%%%
\vspace{2cm}

Print the matrices $\mat{z}_1$ and $\mat{z}_2$ as well as the results of your computations. What do you observe ? How do your results relate to the results you got in the first sub-task ?
\color{blue} \\[1ex]

\begin{python}
print("Z1:")
print(z1, end='\n\n')
print("Z2:")
print(z2)
\end{python}
\mat{z1} & = 
\begin{bmatrix} 
3 & -4 \\ 
4 &  3 
\end{bmatrix} \\ \\
\mat{z2} & = 
\begin{bmatrix} 
 2 & 2 \\ 
-2 & 2 
\end{bmatrix} \\

$z_1 + z_2$ =  5 + 2j 	\\
$z_1 \cdot z_2$ =  14 + 2j	\\	
$z_1^*$   =  3 - 4j	\\
$\lvert z_1 \rvert$ =  5.0	\\ \\	
\mat{$\mat{z}_1 + \mat{z}_2$} & = 
\begin{bmatrix} 
5. & -2. \\ 
2. &  5.
\end{bmatrix} \\ \\

\mat{$\mat{z}_1 \cdot \mat{z}_2$} & = 
\begin{bmatrix} 
14. & -2. \\ 
2. &  14.
\end{bmatrix} \\ \\

\mat{$\trn{\mat{z}_1}$} & = 
\begin{bmatrix} 
3. & 4. \\ 
-4. &  3.
\end{bmatrix} \\ \\

$\sqrt{\det{\mat{z}_1}}$ = 5.000000000000001 \\ \\
%%%%%
%%%%%
%%%%% enter your discussion here
%%%%%
%%%%%
We observe that we can represent a complex number a+bi as matrix $$\begin{bmatrix} a & -b \\ b & a \end{bmatrix}$$ and then:
\begin{description}
  \item[$\bullet$] addition of complex numbers corresponds to their matrix addition
  \item[$\bullet$] multiplication of complex numbers corresponds to their matrix multiplication
  \item[$\bullet$] complex conjugate of a complex number corresponds to transpose of its matrix
  \item[$\bullet$] absolute value of a complex number corresponds to square root of the determinant of its matrix
\end{description}

%%%%%
%%%%%
%%%%%
%%%%%
%%%%%
\color{black}
\newpage





\subsection*{bonus [10 points]}
Just as the complex numbers $z \in \mathbb{C}$, the quaternions $q \in \mathbb{H}$ can be represented in terms of matrices, too. Here, we actually have to choices: 

\begin{enumerate}
\item either, we may introduce certain $4 \times 4$ real valued matrices $\mat{1}, \mat{i}, \mat{j}, \mat{k}$ which represent the unit quaternions $1, i, j, k$ so that a quaternion $\mat{q} = a \, \mat{1} + b \, \mat{i} + c \, \mat{j} + d \, \mat{k}$ is a matrix $\mat{q} \in \mathbb{R}^{4 \times 4}$. Do you have an idea how the ``basis'' matrices $\mat{1}$, $\mat{i}$, $\mat{j}$, and $\mat{k}$ could look like? If so, implement them in Numpy and compute the product $\mat{i} \mat{j} \mat{k}$.
%%%%%
%%%%%
%%%%% enter your code into the following environment
%%%%%
%%%%%
\begin{python}
# Bonus-1
identity = np.identity(4)
i = np.array([[0, -1, 0, 0], [1, 0, 0, 0], [0, 0, 0, -1], [0, 0, 1, 0]])
j = np.array([[0, 0, -1, 0], [0, 0, 0, 1], [1, 0, 0, 0], [0, -1, 0, 0]])
k = np.array([[0, 0, 0, -1], [0, 0, -1, 0], [0, 1, 0, 0], [1, 0, 0, 0]])
print("i*j*k:")
print(i @ j @ k)
\end{python}
%%%%%
%%%%%
%%%%%
%%%%%
%%%%%
\item or, we may introduce certain $2 \times 2$ complex valued matrices $\mat{1}, \mat{i}, \mat{j}, \mat{k}$ which represent the unit quaternions $1, i, j, k$ so that a quaternion $\mat{q} = a \, \mat{1} + b \, \mat{i} + c \, \mat{j} + d \, \mat{k}$ is a matrix $\mat{q} \in \mathbb{C}^{2 \times 2}$. Do you have an idea how the complex valued ``basis'' matrices $\mat{1}$, $\mat{i}$, $\mat{j}$, and $\mat{k}$ could look like? If so, implement them in Numpy and compute the product $\mat{i} \mat{j} \mat{k}$.
%%%%%
%%%%%
%%%%% enter your code into the following environment
%%%%%
%%%%%
\begin{python}
# Bonus-2
identity = np.identity(2)
i = np.array([[complex(0, 1), 0], [0, complex(0, -1)]])
j = np.array([[0, 1], [-1, 0]])
k = np.array([[0, complex(0, 1)], [complex(0, 1), 0]])
print("i*j*k:")
print(i @ j @ k)
\end{python}
%%%%%
%%%%%
%%%%%
%%%%%
%%%%%
\end{enumerate}
Print your results and discuss what you observe.
\color{blue} \\[1ex]
For 1: \\
\mat{$\mat{i} \mat{j} \mat{k}$} & = 
\begin{bmatrix} 
-1 & 0 & 0 & 0 \\ 
 0 & -1 & 0 & 0 \\ 
0 & 0 & -1 & 0 \\ 
0 & 0 & 0 & -1 \\ 
\end{bmatrix} \\ \\ \\
For 2: \\
\mat{$\mat{i} \mat{j} \mat{k}$} & = 
\begin{bmatrix} 
-1.+0.j &  0.+0.j \\ 
 0.+0.j & -1.+0.j
\end{bmatrix} \\ \\
%%%%%
%%%%%
%%%%% enter your discussion here
%%%%%
%%%%%
%%%%%
%%%%%
%%%%%
%%%%%
%%%%%
\color{black}

